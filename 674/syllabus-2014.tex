\begin{filecontents}{syllabus-2014.bib}
@book{Ca07,
  title =	 {Methods for Applied Macroeconomic Research},
  author =	 {Canova, Fabio},
  year =	 2007,
  publisher =	 {Princeton University Press}
}

@book{DD11,
  title =	 {Structural Macroeconometrics},
  author =	 {DeJong, David N and Dave, Chetan},
  year =	 2011,
  edition =	 {second},
  publisher =	 {Princeton University Press}
}

@Article{FR07,
  title =	 {{ABC}s (and {D}s) of Understanding {VAR}s},
  author =	 {Fern{\'a}ndez-Villaverde, Jes{\'u}s and
                  Rubio-Ram{\'\i}rez, Juan F. and Sargent, Thomas
                  J. and Watson, Mark W.},
  journal =	 {American Economic Review, Papers and Proceedings},
  pages =	 {1021-1026},
  month =	 {June},
  year =	 2007,
}

@Book{Ha94,
  author =	 {James D. Hamilton},
  title =	 {Time Series Analysis},
  publisher =	 {Princeton University Press},
  year =	 1994
}

@InCollection{Ki13,
  author =	 {Lutz Kilian},
  title =	 {Structural {V}ector {A}utoregressions},
  booktitle =	 {Handbook of Research Methods and Applications in
                  Empirical Macroeconomics},
  publisher =	 {Edward Elgar Publishing},
  year =	 2013,
  editor =	 {Hashimzade, Nigar and Thornton, Michael A.},
  pages =	 {515-554}
}

@Book{Lu06,
  author =	 {Helmut L{\"u}tkepohl},
  title =	 {New Introduction to Multiple Time Series Analysis},
  publisher =	 {Springer},
  year =	 2006
}

@inproceedings{Lu76,
  title =	 {Econometric Policy Evaluation: A Critique},
  author =	 {Lucas Jr, Robert E.},
  booktitle =	 {Carnegie-Rochester Conference Series on Public
                  Policy},
  volume =	 1,
  pages =	 {19-46},
  year =	 1976,
  organization = {Elsevier}
}

@Misc{MS08,
  author =	 {Anna Mikusheva},
  title =	 {Lecture Notes for Time Series Analysis ({MIT}
                  14.384)},
  note =	 {Available at
                  \url{http://ocw.mit.edu/courses/economics/14-384-time-series-analysis-fall-2013}. Licensed
                  under Creative Commons BY-NC-SA},
  year =	 2013
}

@article{No11,
  title =	 {The Prize in Economic Sciences 2011 --- Advanced
                  Information},
  author =	 {{Nobel Prize Committee}},
  year =	 2011,
  journal =	 {Nobelprize.org},
  note =	 {Available at
                  \url{http://www.nobelprize.org/nobel_prizes/economic-sciences/laureates/2011/advanced-economicsciences2011.pdf}}
}

@Misc{SW08,
  author =	 {James Stock and Mark Watson},
  title =	 {{NBER} {S}ummer {I}nstitute {L}ectures: What's New
                  in Econometrics --- Time Series},
  howpublished = {Video and slides available at
                  \url{http://www.nber.org/minicourse_2008.html}},
  month =	 {July},
  year =	 2008
}

@Article{Si12,
  author =	 {Christopher A. Sims},
  title =	 {Statistical Modeling of Monetary Policy and Its
                  Effects},
  journal =	 {American Economic Review},
  year =	 2012,
  volume =	 102,
  number =	 4,
  pages =	 {1187-1205},
  month =	 {June},
}

@Article{Si80,
  title =	 {Macroeconomics and Reality},
  author =	 {Sims, Christopher A.},
  journal =	 {Econometrica},
  volume =	 48,
  number =	 1,
  pages =	 {1-48},
  year =	 1980
}
\end{filecontents}

\documentclass[nofonts,nols]{tufte-handout}
\title[Econ 674 syllabus]{Econ 674, PhD Macroeconometrics}
\author{Helle Bunzel and Gray Calhoun}
\date{Fall semester, 2014}

\morefloats
\usepackage{
  booktabs,
  enumitem,
  fancyvrb,
  fancyhdr,
  marginfix,
  tabularx,
  url,
  verbatim,
  xcolor,
}
\usepackage[T1]{fontenc}
\usepackage[utf8x]{inputenc}
\usepackage[bitstream-charter]{mathdesign}
\usepackage[letterspace=35]{microtype}
\usepackage[osf]{sourcecodepro}
\usepackage{sourcesanspro}

% Change to ragged-right without inline math breaks; this helps avoid
% awkward line breaks or extensions into the right margin in math or
% code.
\RaggedRight
\relpenalty=10000
\binoppenalty=10000
\setlength{\parindent}{1em}

\renewcommand{\smallcaps}[1]{\allcaps{#1}}
\renewcommand{\allcaps}[1]{\textls{\MakeUppercase{#1}}}

\renewcommand{\FancyVerbFormatLine}[1]{\hspace{\leftmargini}#1}
\fvset{frame=leftline,rulecolor=\color{lightgray}}

\urlstyle{same}
\DeclareUrlCommand\url{%
\def\UrlLeft{\guilsinglleft}%
\def\UrlRight{\guilsinglright}%
\urlstyle{same}}

\DisableLigatures{family = tt*}
\frenchspacing

\newcommand{\email}[1]{\href{mailto:#1}{\nolinkurl{#1}}}
\newcommand{\DSGE}{\allcaps{DSGE}}
\newcommand{\OH}{\allcaps{OH}}
\newcommand{\TA}{\allcaps{TA}}
\newcommand{\VAR}{\allcaps{VAR}}

\renewcommand{\labelitemi}{{$\circ$}}
\renewcommand{\labelitemii}{\footnotesize$\circ$}
\renewcommand{\labelitemiii}{\textperiodcentered}
\renewcommand{\labelitemiv}{\footnotesize\textperiodcentered}

\begin{document}
\maketitle

\begin{table*}[h]
\begin{tabularx}{\textwidth}{rXXX}
  \toprule
         & Gray Calhoun         & Helle Bunzel        & Pan Liu (\TA)        \\
  \midrule
  email  & gcalhoun@iastate.edu & hbunzel@iastate.edu & panliu@iastate.edu   \\
  office & 467 Heady            & 373 Heady           & 280B Heady           \\
  \OH    & Tu 2--3:15           & \allcaps{TBD}       & \allcaps{MW} 2--3:30 \\
  \bottomrule
\end{tabularx}
\caption{Instructor and TA contact information.}
\end{table*}

\noindent%
This is the syllabus for the first part of Econ 674, a
macroeconometrics elective.  If you have questions about the course
material, the best times to address them are in the scheduled class
meetings or during office hours. We can probably resolve questions or
concerns about the course administration over email, but if you have
urgent questions please stop by my office.

The required textbook for this class is Jim Hamilton's \emph{Time
  Series Analysis},\cite{Ha94} which is quite dated in some areas but
is still reasonably comprehensive. Other recommended resources are
\begin{itemize}
\item Helmut L{\"u}tkepohl's \emph{New Introduction to Multiple Time
    Series Analysis},\cite{Lu06} which is an extremely thorough
  treatment of \VAR s and cointegration;
\item Fabio Canova's \emph{Methods for Applied Macroeconomic
    Research},\cite{Ca07} focusing primarily on structural estimation;
\item The lecture notes for Anna Mikusheva's time series class at
  \allcaps{MIT};\cite{MS08}
\item The \allcaps{NBER} Summer Institute lectures given on time
  series by Jim Stock and Mark Watson.\cite{SW08} (Videos and slides
  are available online.)
\end{itemize}
There are a few more papers listed on the syllabus and many more
references on each topic will be given in the lectures.
Additional material may be posted online as well, at
\begin{itemize}
\item \url{http://www.econ.iastate.edu/~gcalhoun/674}, when Gray is
  teaching;
\item Blackboard Learn, \url{https://bb.its.iastate.edu}, when Helle
  is teaching.
\end{itemize}

You can use any software that you'd like for this class.
Matlab is the standard software package in macroeconomics and you've
already used R in the first-year econometric sequence, but I'd
encourage you to try out Julia%
\footnote{Available at \url{http://julialang.org}.} %
--- it's syntax is very similar to
Matlab's, but the language is much better designed.%
\footnote{The similarity of the syntax means that a lot of Matlab code
  will run almost unmodified.} %
If you want to learn a more general-purpose programing language,
Python is a good choice.%
\footnote{Python here means \emph{SciPy}. See
  \url{http://scipy.org}. And you should use Python 3 if you go this
  route, not Python 2. Version 3.5 is coming soon and will introduce
  native infix matrix operators, which will help a lot with code
  readability.} %
All of these languages are free and open-source (except Matlab) and
have been designed for scientific and statistical computing. See the
course webpage for links and additional details.

\section{Grading}
Your grade will be determined by an oral exam and an original research
paper --- each count for half of the grade. The exam will be scheduled
to take place during week 15 and will take each student about 20
minutes. Students will take the test individually. We will discuss
specific details of this test in class and you will be given a list of
topics to prepare before the Thanksgiving break.

For the paper, you will be required to submit a proposal and a first
draft during the semester, and Helle and I will discuss your proposal
with you after you submit it. Your course grade will be listed as
``incomplete'' until you turn in the paper, and the quality of the
proposal and the first draft will factor into your grade on the paper.

\begin{table*}[t]
\begin{tabularx}{\textwidth}{lrX}
  \toprule
  Lecture topic                                                     & Date & Background reading                                             \\
  \midrule
  Introduction                                                      & 8/26 & \citep{Lu76} \citep{No11} \citep{Si12}                         \\
  Basic concepts in time series                                     & 8/28 & \citep{Ha94} Ch 3 \& 4                                         \\
  Estimating ARMA models                                            & 9/02 & \citep{Ha94} Ch 5, 7 \& 8                                      \\
  Structural \VAR s                                                 & 9/04 & \citep{Ha94} Ch 10 \& 11                                       \\
  Bayesian estimation of \VAR s                                     & 9/09 & \citep{Ha94} Ch 12                                             \\
  Partial identification and inference                              & 9/11 & \citep{Ki13}                                                   \\
                                                                                                                                            \\
  \multicolumn{2}{r}{Part II (Helle Bunzel) \hfill 9/16 -- 10/30}   & Heteroskedasticity, bootstrap, unit roots, and structural breaks      \\
  \multicolumn{2}{r}{Part III (Gray Calhoun) \hfill 11/04 -- 12/04} & Cointegration, \allcaps{DSGE} models, and forecasting as time permits \\
                                                                                                                                            \\
  Assignments                                                                                                                               \\
  \cmidrule{1-2}
  Paper proposal due                                                & 10/23                                                                 \\
  \multicolumn{2}{r}{Oral exam \hfill 12/08 -- 12/12}                                                                                       \\
  First draft of paper due                                          & 12/19                                                                 \\
  Final draft of paper due                                          & 1/30                                                                  \\
  \bottomrule
\end{tabularx}
\caption{Lecture outline for the first part of the class; tentative
  topics to be covered later in the semester; and important
  deadlines.}
\end{table*}

\bibliographystyle{alpha}
\bibliography{syllabus-2014}

\section{License and copyright}
To the extent possible under law, Gray Calhoun, the author, has waived
all copyright and related or neighboring rights to this
document. Anyone is free to reuse part or all of this syllabus to
teach a similar class, or for any other purpose. You can download the
LaTeX source code for this file from the course webpage,
\url{https://www.econ.iastate.edu/~gcalhoun/674}.

\newpage

\section{University policies}

The following policies apply to \emph{every} course at Iowa State
University. They are listed here for your convenience and reference.

\subsection*{Academic dishonesty}

The class will follow Iowa State University's policy on academic
dishonesty.  Anyone suspected of academic dishonesty will be reported
to the Dean of Students Office,
\url{http://www.dso.iastate.edu/ja/academic/misconduct.html}.

\subsection*{Disability accommodation}

This material can be provided to you in alternative format. Anyone who
anticipates difficulties with the content or format of the course due
to a physical or learning disability should see me immediately in
order to work out a plan. You may also want to contact the Disability
Resources (DR) office, located on the main floor of the Student
Services Building, Room 1076 or call them at 515-294-7220.

\subsection*{Dead week}

For academic programs, the last week of classes is considered to be a
normal week in the semester except that in developing their syllabi
faculty shall consider the following guidelines:

\begin{itemize}
\item Mandatory final examinations in any course may not be given
  during Dead Week except for laboratory courses and for those classes
  meeting once a week only and for which there is no contact during
  the normal final exam week. Take home final exams and small quizzes
  are generally acceptable. (For example, quizzes worth no more than
  10 percent of the final grade and/or that cover no more than
  one-fourth of assigned reading material in the course could be
  given.)
\item Major course assignments should be assigned prior to Dead Week
  (major assignments include major research papers, projects,
  etc.). Any modifications to assignments should be made in a timely
  fashion to give students adequate time to complete the assignments.
\item Major course assignments should be due no later than the Friday
  prior to Dead Week. Exceptions include class presentations by
  students, semester-long projects such as a design project in lieu of
  a final, and extensions of the deadline requested by students.
\end{itemize}

\subsection*{Harassment and discrimination}

Iowa State University strives to maintain our campus as a place of
work and study for faculty, staff, and students that is free of all
forms of prohibited discrimination and harassment based upon race,
ethnicity, sex (including sexual assault), pregnancy, color, religion,
national origin, physical or mental disability, age, marital status,
sexual orientation, gender identity, genetic information, or status as
a U.S. veteran. Any student who has concerns about such behavior
should contact his/her instructor, Student Assistance at 515-294-1020,
or the Office of Equal Opportunity and Compliance at 515-294-7612.

\subsection*{Religious accommodation}

If an academic or work requirement conflicts with your religious
practices and/or observances, you may request reasonable
accommodations. Your request must be in writing, and your instructor
or supervisor will review the request.  You or your instructor may
also seek assistance from the Dean of Students Office or the Office of
Equal Opportunity and Compliance.

\subsection*{Contact information}

If you feel that any of your rights as a student have been violated,
please email \url{academicissues@iastate.edu}.

\end{document}
